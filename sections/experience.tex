%-------------------------------------------------------------------------------
%    SECTION TITLE
%-------------------------------------------------------------------------------
\cvsection{Experience}


%-------------------------------------------------------------------------------
%    CONTENT
%-------------------------------------------------------------------------------
\begin{cventries}
\cvsubsection{Cognizant Technology Solutions}

  \cventry
    {Associate} % Job title
    {Lincoln Financial Group} % Organization
    {Chennai, India} % Location
    {July 2018 - November 2018} % Date(s)
    {
        Lead of 3 member team and responsible for deliverables in GP applications
        \newline
        \textbf{Technology} - Visual Basic, Java,Unix,SQL \newline
        \textbf{Tools} - IntelliJ, Selenium, Sikuli, Oracle
    }

  \cventry
    {Programmer Analyst} % Job title
    {Lincoln Financial Group} % Organization
    {Chennai, India} % Location
    {July 2017 - July 2018} % Date(s)
    {
        Worked on applications that handles coverage, processes and delivery to end users. It also involves various processes which makes use of frameworks like Struts, Scripting languages and PL/SQL.
        \newline
        Automated critical health check applications using selenium and sikuli scripts\newline
        \textbf{Technology} - Visual Basic, Java, SQL \newline
        \textbf{Tools} - IntelliJ, Selenium, Sikuli, Oracle
    }

  \cventry
    {Programmer Analyst, MetLife Insurance} % Job title
    {Fund Expansion Re-Engineering (FER)} % Organization
    {Coimbatore \& Chennai, India} % Location
    {Feb. - June 2016 \& Jan. - May 2017} % Date(s)
    {
        Typical day includes, performing data validation and data entry for the new set of changes from our product team for new funds. I would then have to compare the Pre-Data and Post-Data reports from our system before promoting the changes to UAT for downstream application team's sign off. While most of these were manual, I \underline{wrote an automation tool} to verify and validate the reports which consumed a lot of manual work.
        \newline
        \textbf{Technology} - J2EE, SQL, Spring, Apache POI \newline
        \textbf{Tools} - RAD 8.5, IBM DB2, WebSphere Application Server (WAS) 8.5, Fluroscent Dev Engine, Maestro Job Scheduler, StarTeam Version Controller
    }

  \cventry
    {Programmer Analyst, MetLife Insurance} % Job title
    {Delayed Settlement Interest (DSI)} % Organization
    {Coimbatore, India} % Location
    {June 2016 - Jan. 2017} % Date(s)
    {
        Took an initiative to \underline{improve the User Experience} of a layout that helped with better navigation in the application. Delivered this feature end to end from UI to schema changes in the database. 
        \newline
        \textbf{Technology} - J2EE, SQL, Spring \newline
        \textbf{Tools} - RAD 8.5, IBM DB2, WebSphere Application Server (WAS) 8.5
    }

  \cventry
    {Programmer Analyst, MetLife Insurance} % Job title
    {Local Market for Automated Renewal (LMAR)} % Organization
    {Coimbatore, India} % Location
    {Nov. 2015 - Feb. 2016} % Date(s)
    {
        Was responsible for writing an end to end documentation for the entire data flow in the system. While the system has been existent for a long time, it had grown out to be complex because of lot of independent services talking to each other. Services were built using Hibernate and Spring. I got exposed to a lot of internal implementation details using these frameworks while trying to understand the work flow for various parts of the system.
    }

  \cventry
    {Programmer Analyst, MetLife Insurance} % Job title
    {Statement of Health (SOH)} % Organization
    {Coimbatore, India} % Location
    {Aug. 2015 - Nov. 2015} % Date(s)
    {
        Was part of the production support team which involved a lot of property file changes for various batch jobs that run on IBM Compute Grid. All the batch jobs bundled the property file along with the code which had an issue of new code deployment for any new property changes. I refactored all the batch jobs to \underline{use the property file from a local machine instead of the EAR archive}. This enabled much faster feedback for all our changes.
        \newline
        \textbf{Technology} - J2EE, SQL, Spring, Grid Computing \newline
        \textbf{Tools} - RAD 8.5, IBM Compute Grid, WebSphere Application Server (WAS) 8.5, Sybase DB, Maestro Job Scheduler, StarTeam Version Controller
    }

  \cventry
    {Programmer Analyst, MetLife Insurance} % Job title
    {Shared Services} % Organization
    {Coimbatore \& Chennai, India} % Location
    {Nov. 2014 - Aug. 2015} % Date(s)
    {
      \begin{cvitems}
        \item{Worked on Query reports, resolved support tickets and reduced quite a lot of backlogs for the team.}
        \item{Analyzed PBI (Problem Investigation) and converted them to PKE (Problem Known Error) and in most cases fixed them too.}
        \item{Providing L2/L3 support which includes IM/NIM/SR activities.}
        \item{Preparing technical specification documents for the problems after performing the Root Cause Analysis (RCA) for certain incidents.}
        \item{Working with different teams (DB, WebOps etc.) to resolve issues}
        \item{Preparing Status report and providing status to Business on weekly basis}
        \item{Support changes in applications post implementation which involves debugging, fixing, and participating in maintenance releases as needed.}
      \end{cvitems}
    }

\end{cventries}
